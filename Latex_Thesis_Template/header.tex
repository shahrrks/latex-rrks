
\usepackage{ifthen}

% ---------------------------------------------------------------
% page layout
% ---------------------------------------------------------------
\usepackage{geometry}
\geometry{twoside,inner=2.5cm, top=2cm, outer=2cm, bottom = 2cm, includefoot, includehead}

% Caption design
\usepackage[bf,sf,nooneline]{caption} % see caption manual for more information
\captionsetup{format=plain}

% ---------------------------------------------------------------
% figures
% ---------------------------------------------------------------
\usepackage{float} 					% force figures to location with [H]
\usepackage[figuresright]{rotating} % rotation of figures
\usepackage{subfigure}				% stack and arange figures
\usepackage{graphics} 				% you now may use jpg
\usepackage{eso-pic}				% include images covering the whole page (see title page)
\usepackage{overpic}				% you can use this package to write over your figure, see manual
\setlength{\subfigcapmargin}{0.5em} % see subfigure package - spacing between labels captions of subfigures

% ---------------------------------------------------------------
% tikz, pstricks style drawing
% ---------------------------------------------------------------
\usepackage{tikz,pgfplots,pgfkeys}	% for drawings (similar to pstricks and Co.)
\usepackage{psfrag} % Only for LaTeX
% ---------------------------------------------------------------
% type face
% ---------------------------------------------------------------
\usepackage[T1]{fontenc}    %% 
% \usepackage[latin1]{inputenc} 
\usepackage[utf8]{inputenc} %%% use utf8x encoding instead latin1

% RUB fonts, if installed
%\renewcommand{\sfdefault}{rubflama}
%\renewcommand{\rmdefault}{rubscala}

% if not you can use these fonts.
% sans serif
\usepackage{helvet}
% serif
\usepackage{lmodern}

\usepackage{pdfpages}

% ---------------------------------------------------------------
% symboles und math
% ---------------------------------------------------------------
\usepackage{amsmath} % math envoirment.
\usepackage{amssymb} 	% more symbols for writing
\usepackage{amsfonts}
\usepackage{fixmath}	% fixes some errors in amsmath
\usepackage{upgreek}
% ---------------------------------------------------------------
% language
% ---------------------------------------------------------------
\usepackage[
%	german,		% Alte deutsche Rechtschreibung
%	ngerman,	% Neue deutsche Rechtschreibung
	english,	% english
%	french,		% francais
]{babel}

% ---------------------------------------------------------------
% footnotes
% ---------------------------------------------------------------
\usepackage{chngcntr} 
\counterwithout{footnote}{chapter} 	% avoids the footnote counter reset after each chapter

% ---------------------------------------------------------------
% literature list
% ---------------------------------------------------------------
% Bibliography using bibtex, include to your document with  \bibliography{bibtex-datei}
% some examples for different bibtex-styles
% \bibliographystyle{plain}
%\bibliographystyle{acm}
\bibliographystyle{plaindin}
%\bibliographystyle{is-abbrv}

% ---------------------------------------------------------------
% Hyperref
% ---------------------------------------------------------------
%% colored links for pdf
%\definecolor{pdfurlcolor}{rgb}{0,0,0.6}
%\definecolor{pdffilecolor}{rgb}{0.7,0,0}
%\definecolor{pdflinkcolor}{rgb}{0,0,0.6}
%\definecolor{pdfcitecolor}{rgb}{0,0,0.6}

% link color for print, black
\definecolor{pdfurlcolor}{rgb}{0,0,0}
\definecolor{pdffilecolor}{rgb}{0,0,0}
\definecolor{pdflinkcolor}{rgb}{0,0,0}
\definecolor{pdfcitecolor}{rgb}{0,0,0}

\usepackage[
   % link color
   colorlinks=true,         % links are colored, not with boxes
   urlcolor=pdfurlcolor,    % \href{...}{...} external (URL)
   filecolor=pdffilecolor,  % \href{...} local file
   linkcolor=pdflinkcolor,  % \ref{...} and \pageref{...}
   citecolor=pdfcitecolor,  % \cite{...}
   % Links
   breaklinks,              % links work with line breaks
   bookmarksnumbered=true,  % enumerate pdf-bookmarks
   bookmarksopen=true, 		% bookmark-view: all subdirectories are open
]{hyperref}

% pdf information
\hypersetup{pdftitle={PDF-LaTeX Vorlage}}


\usetikzlibrary{shapes.multipart} % For rounded rectangles

% Define a new command for including rounded images
\newcommand{\roundedincludegraphics}[2][]{%
    \tikz\node[inner sep=0, rounded corners, outer sep=0] {\includegraphics[#1]{#2}};%
}

% ---------------------------------------------------------------
% head and foot
% ---------------------------------------------------------------
\usepackage[headsepline=.4pt]{scrlayer-scrpage}	% package for headings
\pagestyle{scrheadings} % define the default page-style for your document
% \setheadsepline{.4pt}	% underline head
% \usepackage[headsepline=.4pt]{scrlayer-scrpage}

\automark[section]{chapter} % defines leftmark and rightmark: \automark[<rightmark>]{<leftmark>}
\lehead{\leftmark} 		% left side, even page number: chapter
\rohead{\rightmark} 	% right side, odd page number: section
\ofoot{\pagemark} 		% page number


% ---------------------------------------------------------------
% misc.
% ---------------------------------------------------------------

% TODONOTES, generates a list of todos, placeholder for figures etc.
% useful package for larger documents. Just have a look at the manual.
\usepackage[
%	disable,
	shadow,colorinlistoftodos,color=green!40
	]{todonotes}


% DEFINE RUB-COLORS
% you can use the RUB colors in Matlab, too. Look at sheldon/user-public/aschasse/RUB_colors/createRubColorset.m
\usepackage{calc}
\definecolor{RUB_blue_0}{RGB}{0,53,96}
\definecolor{RUB_blue_1}{RGB}{23,82,125}
\definecolor{RUB_blue_2}{RGB}{80,117,155}
\definecolor{RUB_blue_3}{RGB}{136,157,188}
\definecolor{RUB_blue_4}{RGB}{193,203,221}

\definecolor{RUB_green_0}{RGB}{148,193,27}
\definecolor{RUB_green_1}{RGB}{174,205,82}
\definecolor{RUB_green_2}{RGB}{196,218,131}
\definecolor{RUB_green_3}{RGB}{217,230,176}
\definecolor{RUB_green_4}{RGB}{236,243,218}

\definecolor{RUB_gray_0}{RGB}{221,221,221}
\definecolor{RUB_gray_1}{RGB}{229,229,229}
\definecolor{RUB_gray_2}{RGB}{237,237,237}
\definecolor{RUB_gray_3}{RGB}{245,245,245}

\definecolor{RUB_orange_0}{RGB}{228,136,65}
\definecolor{RUB_orange_1}{RGB}{232,154,84}
\definecolor{RUB_orange_2}{RGB}{239,179,113}
\definecolor{RUB_orange_3}{RGB}{243,188,132}
\definecolor{RUB_orange_4}{RGB}{250,206,167}

\definecolor{RUB_red_0}{RGB}{198,77,50}
\definecolor{RUB_red_1}{RGB}{219,103,78}
\definecolor{RUB_red_2}{RGB}{235,134,111}
\definecolor{RUB_red_3}{RGB}{248,157,143}
\definecolor{RUB_red_4}{RGB}{252,189,179}

% ---------------------------------------------------------------
% own commands
% ---------------------------------------------------------------
% e.g.
\newcommand{\EW}[1]{\text{E}\left\lbrace #1 \right\rbrace } % Expectation


%%% Own new Commands %%%% 
\newcommand{\AUTHOR}{Ravi Rahul Kumar Shah}
\newcommand{\SUPERVISOR}{M. Sc. Luca Becker}
\newcommand{\PROF}{Prof. Dr.-Ing. Rainer Martin}
\newcommand{\TITLE}{Information Bottleneck Based Privacy-preserving Feature Extraction Using a Trainable Acoustic Frontend}


% ---------------------------------------------------------------
% \usepackage{glossaries}
% \usepackage[automake]{glossaries-extra}
% \preto\section{\glsresetall}
% \setabbreviationstyle[acronym]{long-short}

\usepackage[withpage]{acronym}
% \usepackage[style=numeric]{biblatex}
% \addbibresource{references.bib}

% \usetikzlibrary{shapes, arrows, positioning, shapes.geometric, fit}
\usepackage{cleveref}

\usepackage{tabularx}
\usepackage{array}
\usepackage{booktabs}
\raggedbottom

% \usetikzlibrary{shapes.geometric, arrows, positioning, fit}

%Hier sind meine new Commands  DEFINITIONS MACROS 
\newcommand{\VERSUCHSNR}{}
\newcommand{\VERSUCHSNAME}{Thesis Title}
\newcommand{\VERSUCHSDATUM}{20.11.2023}
\newcommand{\PROTOKOLLDATUM}{\today}

\newcommand{\VerfasserEINS}{Ravi Shah}
\newcommand{\MatNoEINS}{108018XXXXXX}



\newcommand{\BETREUER}{Lucas Bilal}
\newcommand{\GRUPPENNR}{Gruppe 1 }

\newcommand{\degr}{^{\circ}}